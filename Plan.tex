\documentclass[10pt,a4paper,final]{scrartcl}
\setlength{\parindent}{0pt}
\usepackage[utf8]{inputenc}
\usepackage[german]{babel}
\usepackage{lmodern}
\usepackage{ulem}
\usepackage{calc}
\usepackage{xcolor}


\newcommand{\raus}[1]{\xout{#1}}
\newcommand{\ueberdenken}[1]{\colorbox{yellow}{\parbox{\textwidth}{??? #1}}}
\newcommand{\unwichtig}[1]{\textcolor{gray}{#1}}
\newcommand{\problem}{\colorbox{red}{\parbox{\widthof{A}}\#\#\#}}
\newcommand{\notiz}[1]{\textcolor{blue}{#1}}

%\thesubsection.
\renewcommand{\labelenumi}{\thesubsection.\theenumi}

\begin{document}



\tableofcontents
\cleardoublepage

\section{Spielmechanik}
\begin{enumerate}
\item spiel soll rundenbasiert laufen, auch über mehrere folgespiele
\item falls ein spiel sich über eine festgelegte rundenzahl zieht, dann tritt der sudden death ein, 
der das ende des spiels näherbringt
\item boni können innerhalb der spiele erworben werden, sodass man einen vorteil im nächsten spiel bekommt
(mehr startguthaben, extrapanzer...)
\item nach festgelegter spieleanzahl wird der sieger festgelegt \notiz{best of 7}
\item jeder spieler fängt an mit einer \raus{citadella}, einem scout und einem bauwagen \notiz{vielleicht insgesamt 5 (auch panzer)}
\item panzer verwenden beim schießen munition, verbrauchen zusätzlich konstant viele aktionspunkte 
\item aktionspunkteverbrauch ist determiniert durch art der munition 
(projektil:a punkte, rakete: b punkte, spezial: c punkte)
\item panzer verbrauchen treibstoff beim fahren, das kostet aktionspunkte und treibstoff
(beides an treibstoffverbrauch festgemacht)
\item jede runde werden aktionspunkte bis einen festgelegten wert aufgefüllt 
der wert kann auf verschiedene weisen erhöht werden
\end{enumerate}

\section{Nachschub}
\begin{enumerate}
\item rucksackinhalte können auf dem gelände liegen gelassen werden oder auch als vorrat abgeworfen werden
(zweites kostet aktionspunkte, liegende päckchen können eingesammelt werden)
\item abgelegte päckchen haben hp und können zerstört werden
\item im falle der zerstörung dieser wird eine explosion ausgelöst. ihr radius und schaden wird durch den 
gesamtpreis ihrer inhalte determiniert \notiz{inhalt explodieren lassen?}
\end{enumerate}

\section{Gebäude}
\begin{enumerate}
\item kommen in 4 typen vor haus, neutraler bunker, fabrik und waffenhändler \notiz{gebäude nicht komplett zerstören(kein delete)}
\item alle können durch fahrzeuge erobert werden, was aber aktionspunkte kostet
\item dazu wird eine variable benutzt, die durch das erobern durch fahrzeuge gefüllt wird 
\item ein fahrzeug kann pro spielzug nur begrentzt viel erobern
\item der variablenwert wird zurückgesetzt, falls das gebäude eine runde lang nicht erobert wurde
\item scouts erobern doppelt so schnell und verbrauchen dafür keine aktionspunkte
\item wenn der wert der variable einen richtwert übersteigt, dann gehört das gebäude den erobernden spieler
\item das gebäude kann pro zeit nur von höchstens einem spieler erobert werden \notiz{Eroberungssperre, nach eroberung (1 runde?)}
\item der besitzer wird durch eine fahne seiner farbe am gebäude festgesetzt, die nach rechts ragt
\item \ueberdenken{der eroberer des gebäudes wird auch durch eine fahne erkannt, die fahne ragt aber nach links}
\item \notiz{besser ist: blinken des gebäudes, welches erobert wird in der farbe des eroberers}
\item jedes fahrzeug kann bei fabrik und waffenhändler einkaufen, soweit noch platz im rucksack
\item jedes gebäude kann repariert, zerstört, erobert und ausgebaut werden
\item sollte das gebäude komplett zerstört werden, kann man dessen trümmer für geld einsammeln
\end{enumerate}

\subsection{Haus}
\begin{enumerate}
\item generiert geld jede runde, der aufenthalt neben deinem \notiz{(jedem?)} haus reduziert den erhaltenen schaden
\notiz{(ist das nicht eh so?)}
\item \ueberdenken{durch aufrüsten erhält der mehr hp, wird schwerer zu erobern, generiert mehr geld
und bietet besseren schutz vor schaden} \notiz{nicht aufrüsten/keine updates}
\end{enumerate}

\subsection{neutraler bunker}
\begin{enumerate}
\item ein panzer im freundlichen bunker ist komplett vor schaden geschützt,  \notiz{durch bunker durchfahren, wenn besetzt?}
\item dazu muss der bunker erst selbst zerstört werden
\item gibt keine überreste bei zerstörung, beschädigt das drinstehende fahrzeug bei zerstörung
\item aufrüsten erhöht hp und deren regeneration pro runde als auch die anzahl der fahrzeuge, 
die von ihn auf einmal geschützt werden können
\item hat wesentlich mehr hp als der von bauwagen gebaute bunker \notiz{aus bunker feuern?}
\end{enumerate}

\subsection{Fabrik}
\begin{enumerate}
\item produziert fahrzeuge, pro runde kann nur ein fahrzeug pro fabrik gebaut werden
\item repariert freundliche fahrzeuge am ende der runde ein wenig \notiz{im umkreis?}
\item frisch produziertes fahrzeug leidet an produktionsschwäche und kommt der runde, in der es gebaut wurde, 
nicht zum einsatz
\item aufrüsten verringert die produktionskosten und gibt weitere fahrzeuge frei als auch erhöht hp, kann mehr reparieren
und wird schwerer zu erobern
\end{enumerate}

\subsection{Waffenhändler}
\begin{enumerate}
\item verkauft waffen, upgrades und konsumierbares an freundliche fahrzeuge, erhöht aktionspunkte ein wenig
\item aufrüsten verringert die produktionskosten und gibt weitere items frei als auch erhöht hp, gibt größeren 
aktionspunktebonus und wird schwerer zu erobern
\end{enumerate}

\section{Fahrzeuge}
\begin{enumerate}
\item \ueberdenken{ein spieler kann beliebig viele fahrzeuge besitzen}
\item jedes hat einen eigenen rucksack, AP sind aber global
\item abschuss der gegnerischen gebäude und fahrzeuge als auch erobern der gegnergebäude erhöht erfahrung des fahrzeugs;
fahrzeug steigt levels auf, wenn genügend erfahrung gesammelt
\item levels erhöhen immer die hp und schaden, levelstufen x und y bescheren zusätzliche boni (x<y)
\item bei zerstörung des fahrzeugs bleiben seine überreste, die für geld und munition gesammelt werden sollen
\item geldbonus ergibt sich aus den geldkosten des fahrzeugs, 
\item munition aus prozentsatz der rucksackmunition des zerstörten fahrzeugs
\item jedes fahrzeug kann bis zu drei kompatible upgrades bei sich installieren
\item es ist nicht möglich die upgrades zu deinstallieren
\item jedes fahrzeug kann pro runde von höchstens drei statusveränderungen betroffen sein, weitere statusveränderungen
zeigen keine wirkung
\item jedes fahrzeug und spieler kann päckchen sammeln
\item jedes fahrzeug kann beliebige güter im rucksack aufbewahren, aber nicht alle nutzen \notiz{mit anderen Tauschen? (items)}
\end{enumerate}

\subsection{Artillerie}
\begin{enumerate}
\item nutzt raketen und spezialmunition 
\item \unwichtig{verbraucht weniger AP per schuss als andere fahrzeuge bei selben waffenklasse, rohr sehr gut verstellbar}
\item \unwichtig{hat erhöhten treibstoffverbrauch} \ueberdenken{ und minimalabschusskraft verschieden von 0}
\item \unwichtig{bei level x : AP verbrauch beim schießen verringert sich noch weiter}
\item \unwichtig{bei level y : verbraucht keine munition, falls standardrakete abgefeuert wird}
\end{enumerate}

\subsection{Panzer}
\begin{enumerate}
\item \unwichtig{nuzt geschosse und raketen, keine besonderen boni}
\item \unwichtig{hat zusammen mit bauwagen größte hp von serienfahrzeugen}
\item rohr nur im kleinen winkel verstellbar, muss gut positioniert werden
\item \unwichtig{bei level x: bekommt weniger schaden von allen waffen}
\item \unwichtig{bei lever y: bekommt dauerhaften hp-und schadensbonus für jedes zerstörte fahrzeug}
\end{enumerate}

\subsection{Scout}
\begin{enumerate}
\item \unwichtig{nutzt geschosse und streugewehr(levelaufstieg)}
\item \unwichtig{sehr mobil durch kleinen treibstoffverbrauch und hohe geschwindigkeit, geländegängig}
\item doppelte eroberungsgeschwindigkeit ohne AP verbrauch
\item sehr geringe hp, raketentreffer meist fatal
\item kann rohr nicht verstellen, schießt geradeaus
\item kann auch durch upgrades keine weiteren munitionsarten abschießen
\item \unwichtig{bei level x: schaltet das streugewehr frei}
\item \unwichtig{bei level y: fahren benötigt keine AP}
\end{enumerate}

\subsection{bauwagen \notiz{derzeit crawler}}
\begin{enumerate}
\item nutzt bautools, geländetool und airstrikes \notiz{andere Fahrzeuge betanken/reparieren?}
\item geländegängig, verbraucht beim fahren kein treibstoff, \ueberdenken{verbraucht aber AP}
\item \ueberdenken{immun gegen negative effekte, wie korrosion oder einfrieren}
\item \ueberdenken{ einziges fahrzeug, was (feindliche) gebäude reparieren und verbessern kann}
\item \unwichtig{bei level x: verringerte reparatur-und aufrüstungskosten der neutralen gebäude}
\item \unwichtig{bei level y: von ihn gebaute gebäude sind verbessert, brückenbau verbraucht keine munition}
\end{enumerate}

\subsection{\raus{Tarnpanzer}}
\begin{enumerate}
\item \raus{nutzt raketen und geschosse}
\item \raus{hat tarnkit als upgrade integriert, hat nur 2 weitere slots für upgrades} \notiz{besonders? (ist ein scout)}
\item \raus{die bewegungen werden nicht dem gegner gezeigt}
\item \raus{für den gegner tarnt sich als ein baum}
\item \raus{tarnung fliegt für die runde auf, falls der schaden in der runde schaden bekommt}
\item \raus{werte und schussverhalten wie bei normalen panzer, deutlich weniger hp}
\item \raus{bei level x: kann in den gegnerischen gebäuden einkaufen}
\item \raus{bei level y: tarnung fliegt nie auf}
\end{enumerate}

\subsection{\raus{Snipercrawler}}
\begin{enumerate}
\item \raus{nutzt geschosse}
\item \raus{direckte treffer machen doppelten schaden}
\item \raus{hat eine zielhilfe als upgrade integriert, hat nur 2 weitere slots für upgrades}
\item \raus{beim schießen werden bäume und gebäude ignoriert  } \notiz{besonders?}
\item \raus{schießt immer mit maximaler stärke}
\item \raus{rohr nur um wenige grad verstellbar}
\item \raus{hp mit artilleriehp vergleichbar}
\item \raus{kann auch durch upgrades keine weiteren munitionsarten abschießen}
\item \raus{bei level x: jeder erfolgreiche gegnertreffer erhöht seinen schaden in dieser runde}
\item \raus{bei level y: bekommt kostenloses tarnupgrade, falls noch platz in upgrades vorhanden}
\end{enumerate}

\subsection{\raus{Citadella}}
\begin{enumerate}
\item \raus{nutzt airstrikes} \notiz{aber wozu dann das Fahrzeug?, stell ich ganz rechts auf die Karte, fertig}
\item \raus{beschränkt auf max eine einheit pro spieler}
\item \raus{sehr, sehr viel hp, regeneriert einen kleinen teil der hp jede runde}
\item \raus{verbraucht beim fahren keinen treibstoff, aber AP; sehr langsam, nicht geländegängig}
\item \raus{generiert geld}
\item \raus{bei deren zerstörung hat der spieler eine festgelegte rundenzahl, um entweder das spiel zu gewinnen oder sich 
citadella zu ersetzen}
\item \raus{kann sich nicht in bunkern verbergen}
\item \raus{hat eine eroberungsgeschwindigkeit von 0}
\item \raus{der gegner sieht ihre hp}
\item \raus{kann keine levels aufsteigen}
\end{enumerate}

\subsection{Sonstiges}
\begin{enumerate}
\item \notiz{Fahrzeuge bei levelup mehr Updateslots?}
\end{enumerate}

\section{Waffen}
\begin{enumerate}
\item \unwichtig{werden in 8 arten eingeteilt: geschoss, bautool, geländetool, streugewehr, rakete, airstrike, spezial und minen}
\item \unwichtig{jede art hat speziefischen AP verbrauch beim abschuss}
\end{enumerate}

\subsection{Geschoss}
\begin{enumerate}
\item geringer schaden, geringe kosten, geringer AP verbrauch
\item kleiner explosionsradius, gut für direckttreffer
\item alle geschosse können bei einem level 1 waffenhändler erworben werden
\item \ueberdenken{wird nicht durch wind beeinflusst}
\end{enumerate}

\subsection{Standardgeschoss}
\begin{enumerate}
\item billigste munition im spiel, eignet sich um scouts auszuschalten oder auf tarnung zu prüfen
\end{enumerate}

\subsection{Explosivgeschoss}
\begin{enumerate}
\item wesentlich mehr schaden als standardgeschoss, \unwichtig{kann gegen alle fahrzeugtypen verwendet werden}
\item \ueberdenken{preis steigt mehrfach bez standardgeschoss}
\end{enumerate}

\subsection{\raus{Reinigungsgeschoss}}
\begin{enumerate}
\item \raus{verringert negative statuseffkte bei treffer eines freundlichen fahrzeugs}
\item \raus{löscht feuer im geringen einschlagsradius}
\end{enumerate}

\subsection{Bautool}
\begin{enumerate}
\item die baupakete werden wie ganz normale waffen gekauft und in den rucksack gelagert\problem
\item "abschuss" des pakets bewirkt bau des gebäudes\problem
\item bauwagengebäude können nicht erobert oder aufgerüstet werden\problem
\item bauwagengebäude hinterlassen keine überreste nach ihrer zerstörung
\item das benutzen dieser waffenklasse gibt erfahrung
\end{enumerate}

\subsection{Bunker}
\begin{enumerate}
\item besitzt die funktionalität des neutralen bunkers, allerdings mit den oben benannten einschränkungen
\item \unwichtig{verbesserte version hat mehr hp und eine hpregeneration pro runde}\problem
\end{enumerate}

\subsection{\raus{Dispencer}}
\begin{enumerate}
\item \raus{heilt alle freundlichen fahrzeuge im bestimmten umkreis ein wenig am ende der runde}
\item \raus{verringert die dauer der negativen statusänderungen der freundlichen fahrzeuge am ende der runde}
\item \raus{wirkungsumkreis durch halbkreis angedeutet}
\item \raus{verbesserte version hat mehr heilkraft und verringert die statuseffekte schneller}
\end{enumerate}

\subsection{Sentrygun}
\begin{enumerate}
\item schießt je nach typ geschosse oder raketen
\item hat keinen rucksack/upgradeslots/statusslots/menubalken
\item man kann durch tastendruck wählen, ob man fahrzeuge oder sentrys verwaltet
\item kann wie ein panzer ins focus genommen werden; kann schießen, aber nicht fahren
\item kann nur standardgeschosse verwenden, kein munitionsverbrauch, kein AP verbrauch
\item hat nur eine begrentzte schussanzahl pro runde
\item verbesserte version hat deutlich mehr schuss pro runde
\end{enumerate}

\subsection{\raus{Geldgenerator}}
\begin{enumerate}
\item \raus{generiert geld am ende jeder runde}
\item \raus{verbesserte version ist deutlich stabiler und generiert mehr geld}
\end{enumerate}

\subsection{Teleporter \notiz{/Tunnel/Tunnelnetzwerk}}
\begin{enumerate}
\item beim benutzen durch freundliches fahrzeug werden AP verbraucht und ein freundlicher teleporter ausgesucht, 
aus dem das fahrzeug erscheinen soll
\item nutzung sinnlos, wenn nur ein teleporter im besitz
\notiz{teleport dauert 1 Runde}
\item \unwichtig{verbesserte version teleportiert für weniger AP und hat mehr hp}\problem
\end{enumerate}

\subsection{Geländetool}
\begin{enumerate}
\item beim ausrüsten dieser waffenklasse wird das abfeuern zwischen normalen fahren und geländetoolnutzung unterscheiden
\item alle geländetools verhalten sich bei ihrer nutzung wie treibstoff beim fahren
\item fahrtrichtung wird nun durch die baurichtung vorgegebenm und nicht durch die karte
\item fahrtrichtungen werden durch die rohrsteuerung gegeben
\end{enumerate}

\subsection{Brücke}
\begin{enumerate}
\item bei geländetoolnutzung: baut eine brücke beim fahren
\item bau erfolgt in fahrtrichtung
\end{enumerate}

\subsection{Graben}
\begin{enumerate}
\item bei geländetoolnutzung: gräbt einen tunnel beim fahren
\item graben erfolgt in fahrtrichtung
\end{enumerate}

\subsection{Streugewehr}
\begin{enumerate}
\item \unwichtig{viel schaden, mittlerer preis, kleiner AP verbrauch (> geschoss)}
\item \ueberdenken{wird nicht durch wind beeinflusst}
\notiz{eine art MG ist besser}
\item \unwichtig{kein bonus bei direckttreffer}
\item schaden und geländezerstörung verringert sich stark mit entfernung zum ziel\problem
\item sehr kurzer projektiltimeout\problem
\end{enumerate}

\subsection{standard Streugewehr}

\subsection{korrosives Streugewehr}
\begin{enumerate}
\item hat denselben schaden wie das standard, gegner bekommt den status korrosion dazu
\item die stärke der korrosion und die anzahl der wirkrunden wird durch den schusschaden festgelegt
\end{enumerate}

\subsection{\raus{elektrisches Streugewehr}}
\begin{enumerate}
\item \raus{deutlich weniger schaden als standard, elektrisiert den gegner}
\item \raus{die länge der elektrisierung wird durch den schaden festgelegt}
\end{enumerate}

\subsection{explosives Streugewehr}
\begin{enumerate}
\item wesentlich mehr schaden als standard
\item etwas teuerer als standard
\item \ueberdenken{rückstoßschaden beim abschuss}
\end{enumerate}

\section{Raketen}
\begin{enumerate}
\item viel schaden, hoher preis, mittlerer AP verbrauch
\item haben einen großen effektradius, direckte trefer nicht nötig
\item stärke der raketeneffekte wird durch den schützenlevel festgelegt
\item es wird ein hohen waffenhändlerlevel gebrauch, um raketen einzukaufen
\item große geländezerstörung
\end{enumerate}

\subsection{Standardrakete}
\begin{enumerate}
\item einzige rakete, die bei level 1 waffenhändler erhältlich ist
\end{enumerate}

\subsection{\raus{Cryorakete}}
\begin{enumerate}
\item \raus{kleiner schaden, mit den explosivgeschoss vergleichbar}
\item \raus{friert fahrzeuge in wirkungsbereich ein}
\end{enumerate}

\subsection{\raus{Neutralisationsrakete}}
\begin{enumerate}
\item \raus{kein schaden}
\item \raus{verringert negative statuseffekte, löscht feuer und heilt fahrzeuge in wirkungsbereich}
\item \raus{wirkungsbereich größer als bei standard}
\end{enumerate}

\subsection{Säurerakete}
\begin{enumerate}
\item kein schaden
\item sehr starker korrosionseffekt auf alle panzer im wirkungsbereich
\end{enumerate}

\subsection{Napalmrakete}
\begin{enumerate}
\item schaden und wirkungsbereich wie bei standard
\item zündet gelände im wirkungsbereich an
\end{enumerate}

\subsection{Schockrakete}
\begin{enumerate}
\item schaden wie bei normaler rakete, größerer wirkungsradius
\item fahrzeuge und gebäude in der reichweite werden elektrisiert
\end{enumerate}

\subsection{Airstrikes}
\begin{enumerate}
\item \unwichtig{hoher schaden, hoher preis, hoher AP verbrauch}
\item raketenbündel, verhalten sich wie raketen
\item anzahl durch schützenlevel determiniert
\item abschuss durch citadella zählt als abschuss mit maximallevel
\item es kann eine stelle auf der karte gewählt werden, in welche etwa die raketen einschlagen sollen
\end{enumerate}

\subsection{standard Airstrike}
\subsection{\raus{neutralisationsairstrike}}
\subsection{säureairstrike}
\subsection{napalmairstrike}
\subsection{schockairstrike}

\subsection{Spezial}
\begin{enumerate}
\item \unwichtig{enorm viel schaden, hoher preis, hoher AP verbrauch} 
\item \unwichtig{besonders starke waffen, die mit einem schlag kampfsituation verändern können}
\item hoher waffenhändlerlevel notwendig, um diese kaufen zu können
\item \unwichtig{sehr großer wirkungsbereich}
\end{enumerate}

\subsection{Nuke}
\begin{enumerate}
\item \unwichtig{BOOOM!}
\item \unwichtig{wo ist das ganze land hin?}
\item brennendes gelände im einschlagsbereich
\end{enumerate}

\subsection{\raus{Reparaturrakete}}
\begin{enumerate}
\item \raus{heilt alle freundlichen fahrzeuge im wirkradius komplett (citadella bekommt nur 25 Prozent heilung)}
\item \raus{hebt alle negativen statuseffekte der freundlichen fahrzeuge auf}
\item \raus{repariert bis zu einem kritischen schaden bei jedem fahrzeug}
\item \raus{alle fahrzeuge im wirkungsbereich bekommen statusveränderung: regeneration}
\item \raus{alle fahrzeuge im wirkungsbereich bekommen statusveränderung: ummunität zu negativen statuseffekten für 3 runden}
\end{enumerate}

\subsection{\raus{finger des todes}}
\begin{enumerate}
\item \raus{macht eine festgesetzte anzahl an schadenspunkten, die durch schützenlevel bestimmt ist}
\item \raus{sucht fahrzeuge neben einschlagebereich, diese erhalten soviel schaden bis die entweder zerstört sind 
oder der schaden ausgeht}
\item \raus{falls nach zerstörung des fahrzeugs noch schaden übrig wird das nächste fahrzeug ausgesucht}
\item \raus{citadella kann nicht als ziel ausgesucht werden}
\end{enumerate}

\subsection{EMP}
\begin{enumerate}
\item \ueberdenken{zerstört alle bauwagengebäude und minen im wirkungsbereich}
\item \ueberdenken{alle
 fahrzeuge in der reichweite werden für sehr viele runden elekrtrisiert}
\end{enumerate}

\subsection{Minen}
\begin{enumerate}
\item variabler schaden, variable kosten, kleiner AP verbrauch\problem
\item \notiz{am besten als Waffe von Baufahrzeug oder/und Scout}
\item sind waffen an sich, jedes fahrzeug kann sie benutzen
\item werden beim drauffahren aktiviert; erkennen nicht, ob freund/feind
\item werden nach rundenende des spielers, der sie gelegt hat, aktiviert
\item können durch schaden entfernt werden
\item allein die verwendung von denen bringt erfahrung (nach kosten)
\item \unwichtig{kaum geländezerstörung}
\item machen immer konstant viel schaden, außer das fahrzeug hat schutz vor schaden\problem
\end{enumerate}

\subsection{standard Mine}
\begin{enumerate}
\item \unwichtig{nicht viel schaden, trotzdem sehr ärgerlich für scouts (3 minen-1 scout)}
\end{enumerate}

\subsection{\raus{Mine}}
\begin{enumerate}
\item \raus{scout tot :)}
\item \raus{explosionsradius einer rakete}
\end{enumerate}

\subsection{\raus{Freezemine}}
\begin{enumerate}
\item \raus{kein schaden}
\item \raus{fahrzeug 2 runden gefroren}
\end{enumerate}

\subsection{\notiz{Elektromine}}

\subsection{Napalmmine}
\begin{enumerate}
\item schaden von normaler mine, zündet gelände an
\end{enumerate}

\subsection{\raus{Napalmmine}}
\begin{enumerate}
\item \raus{siehe mine +, zündet gelände im sehr großen radius an}
\end{enumerate}

\section{Upgrades}
\begin{enumerate}
\item bis zu drei an einem fahrzeug auf einmal
\item ausrüsten möglich, falls platz vorhanden
\item unmöglich herunterzunehmen
\item man kann nicht mehrere gleiche upgrades auf einmal haben
\item können bei waffenhändler mit hohem level gekauft werden
\end{enumerate}

\subsection{Zielhilfe}
\begin{enumerate}
\item entlang der karte wird ein strahl gezeichnet, der vom panzerrohr ausgeht und dessen winkel hat
\item \notiz{Zielhilfe mit beschränkter länge ist besser}
\end{enumerate}

\subsection{Zusatzplatinen}
\begin{enumerate}
\item empfangener schaden wird verringet \notiz{mehr HP besser}
\end{enumerate}

\subsection{Vorrat}
\begin{enumerate}
\item standardmunition wird beim schießen nicht verbraucht \notiz{mehr Rucksack?}
\end{enumerate}

\subsection{Tarnkit}
\begin{enumerate}
\item aktionen des fahrzeugs für gegner nicht sichtbar, während seinen zug sieht der gegner einen baum statt des fahrzeugs bei Aktion nächste Runde sichtbar (Panzer)
\end{enumerate}

\subsection{Mechaniker}
\begin{enumerate}
\item fahrzeug regeneriert einen anteil an gesundheit jedes rundenende
\end{enumerate}

\subsection{Commando}
\begin{enumerate}
\item besitz des fahrzeugs mit commando erhöht AP
\end{enumerate}

\subsection{\raus{quick revive}}
\begin{enumerate}
\item \raus{bei zerstörung des fahrzeugs wird das fahrzeug nicht zerstört, sondern heilt sich zu vollem hp, }
\item \raus{quick revive wird zerstört}
\end{enumerate}

\section{Konsumierbares}
\begin{enumerate}
\item wirkt entweder sofort oder vorübergehend
\item vorübergehende effekte werden als statusänderungen eingetragen, treffen nicht ein, 
falls das fahrzeug schon 3 statusänderungen hat
\item können bei waffenhändler erworben werden
\end{enumerate}

\subsection{Reparaturkit}
\begin{enumerate}
\item stellt gesundheit wieder her \notiz{am ende der Runde?}
\end{enumerate}

\subsection{\raus{invulncanteen \notiz{/ unverwundbar}}}
\begin{enumerate}
\item \raus{das fahrzeug ist diese runde unverwundbar}
\item \raus{fahrzeug dennoch zerstört, falls es ins wasser fällt}
\end{enumerate}

\subsection{AP schub}
\begin{enumerate}
\item erhöht die AP für diese runde etwas
\end{enumerate}

\subsection{\raus{Teleport}}
\begin{enumerate}
\item \raus{es wird analog airstrike ein punkt auf der karte gewählt, dort wird das fahrzeug verlagert}
\end{enumerate}

\section{Statusänderungen}

\subsection{korrosion}
\begin{enumerate}
\item macht am ende der runde schaden an fahrzeug
\end{enumerate}

\subsection{elektrisiert}
\begin{enumerate}
\item fahrzeug kann nicht rohr verstellen items verwenden oder waffen wechseln \notiz{fahren?}
\end{enumerate}

\subsection{gefroren}
\begin{enumerate}
\item kann nichts machen und bekommt mehr schaden
\end{enumerate}

\subsection{\notiz{brennend}}


\section{Extraanforderungen}
\subsection{gelände soll vielfältiger werden }
\begin{enumerate}
\item brennendes gelände (macht schaden pro \unwichtig{zeit} \notiz{Runde} wenn ein fahrzeug/gebäude in spielzug des besitzers drauf befindet)
\item sumpf (erhöhter treibstoffverbrauch, langsamere bewegungsgeschwindigkeit)
\item \unwichtig{silizium (treffer generiert splitter, die geld bringen)}

\item \notiz{nach einem negativ Status eine gewisse Statussperre (also erst nach paar runden wieder möglich) für diesen}
\end{enumerate}

\section{weitere Ideen}
\begin{enumerate}
\item \notiz{woher kommt Treibstoff (kaufen, produzieren (Fabrik)) ?}
\item \notiz{wenn Baufahrzeug an Bebäude/Baustelle vorbeifährt, Umriss des Gebäudes zeichnen}
\item \notiz{airstrike nur in gewissem Umkreis des Panzers erlauben}
\item \notiz{Taste für 'alle Gebaeudeumrisse zeichnen'}
\item \notiz{maximal Fuel pro Panzer}
\item \notiz{Tunnelmenge begrenzen}
\item \notiz{erobern macht nach plan keinen Sinn}
\item \notiz{garkeine Geländezerstörung}
\item \notiz{erobern: 1 Runde am Gebäude (prüfen bei Beginn der eigenen Runde)}
\item \notiz{Fahrzeug: maximale Reichweite pro Runde (Radius oder Sprit)}

\item \notiz{Geschütztürme begrenzen (max) 20? 10?}
\item \notiz{Bunker begrenzen 10? 5?}
\item \notiz{Häuser credits generierung logarithmisch}
\item \notiz{feuerschaden erfolgt über Runden}
\end{enumerate}


\end{document}